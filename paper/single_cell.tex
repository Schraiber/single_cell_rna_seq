\documentclass[11pt, oneside]{article}   	% use "amsart" instead of "article" for AMSLaTeX format
\usepackage{geometry}                		% See geometry.pdf to learn the layout options. There are lots.
\geometry{letterpaper}                   		% ... or a4paper or a5paper or ... 
%\geometry{landscape}                		% Activate for rotated page geometry
%\usepackage[parfill]{parskip}    		% Activate to begin paragraphs with an empty line rather than an indent
\usepackage{graphicx}				% Use pdf, png, jpg, or eps§ with pdflatex; use eps in DVI mode
								% TeX will automatically convert eps --> pdf in pdflatex		
\usepackage{amssymb}
\usepackage{amsmath}

%SetFonts

%SetFonts


\title{Non-parametric inference of steady state RNA distributions from single cell transcriptomic data}
\author{Joshua G. Schraiber}
\date{Started on March 17, 2017. Compiled on \today}							% Activate to display a given date or no date

\begin{document}
\maketitle
\section{Basic model}
Suppose we have a genome with $G$ genes in it, each of which produces mRNA transcripts in each of $C$ different cells. We assume that each gene $j$ is expressed at some level $e_{ij}$ in cell $i$, with $e_{ij} \in \mathbb{N}$. We further assume that there exists a steady-state distribution of expression levels for each gene $j$, $\mathbb{R}_j$. Because , we can represent $\mathbb{R}_j$ as a sum of delta masses, say
\[
\mathbb{R}_j = \sum_{k=0}^\infty \pi_{jk}\delta_k
\]
where $\delta_k$ is the delta mass at $k$ and $\pi_{jk}$ is the probability that there are $k$ transcripts of gene $j$ in a cell.

We assume we have performed single-cell transcriptomics on each cell. Thus, for each cell $i$, we have $N_i$ total sequencing reads, and for each gene $j$, we observe the counts $r_{ij}$ of reads of that gene in that cell. Note that $r_{ij}$ can be thought of as a noisy proxy for $e_{ij}$, and we specifically assume that the the $r_{ij}$ are obtained by multinomial sampling from the $e_{ij}$. Thus, our full model is
\begin{align*}
e_{ij} &\sim \mathbb{R}_j \\
\{r_{ij}, 1 \leq j \leq G\} | \{e_{ij}, 1 \leq j \leq G\} &\sim \text{Multinomial}\left(N_i; \frac{e_{i1}}{\sum_j{e_{ij}}},\frac{e_{i2}}{\sum_j{e_{ij}}},\ldots, \frac{e_{iG}}{\sum_j{e_{ij}}}\right).
\end{align*}

In essence, we would like to infer the $\pi_{jk}$ from the $r_{ij}$. Note that if we had direct access to the $e_{ij}$ that would be easy: you can simply estimate
\[
\hat{\pi}_{jk} = \frac{\sum_{i=1}^C \mathbb{I}\{e_{ij} = k\}}{C}.
\]

This suggests an EM algorithm. However, an proper EM algorithm would be very difficult, because of the fact that the read counts of every gene in a cell depends on the read count of every other gene in that cell. Instead, we propose an approximate EM algorithm as follows.

We need to compute the posterior probability, given the current estimates of $\pi_{jk}$, that $e_{ig} = k$, this is given by
\[
\mathbb{P}(e_{ij} = k | r_{ij}, \{\pi_{jk}\}) \propto \mathbb{P}(r_{ij} | e_{ij} = k)\pi_{jk}
\]

To do so, we first approximate the distribution of read counts given all of the $e_{ij}$ by the marginal binomial distributions, i.e.
\[
\mathbb{P}(r_{ij} | e_{ij} = k, \{e_{ig}, g \neq j\}) = \binom{N_i}{r_{ij}} \left(\frac{k}{\sum_{g \neq j} e_{ig} + k}\right)^{r_{ij}}\left(1-\frac{k}{\sum_{g \neq j} e_{ig} + k}\right)^{N_i - r_{ij}}.
\]

We then marginalize over all the $e_{ig}$ for $g \neq j$. Doing that exactly would be difficult. Thus, we proceed with an approximation. Letting $T_{jk} = \sum_{g \neq j }e_{ig} + k$, we can Taylor exapnd around $\mathbb{E}(T_{jk})$, 
\[
\mathbb{E}\left(\binom{N_i}{r_{ij}} \left(\frac{k}{T_{jk}}\right)^{r_{ij}}\left(1-\frac{k}{T_{jk}}\right)^{N_i - r_{ij}}\right) \approx \binom{N_i}{r_{ij}} \left(\frac{k}{\mathbb{E}(T_{jk})} \right)^{r_{ij}}\left(1 - \frac{k}{\mathbb{E}(T_{jk})} \right)^{N_i - r_{ij}}\left(1 + C_k \text{Var}(T_{jk})\right)
\]
where $C_k$ is a really ugly constant. Note that given the $\pi_{jk}$, we can compute $\mathbb{E}(e_{ij}) = \sum_{k=0}^\infty k \pi_{jk}$ and $\mathbb{E}(e_{ij}^2) = \sum_{k=0}^\infty k^2 \pi_{jk}$. Thus, if we let $T = \sum_g e_{ig}$ (i.e. without fixing any genes to a specific value), we can compute $\mathbb{E}(T_{jk}) = \mathbb{E}(T) - \mathbb{E}(e_{ij})+k$ and $\text{var}(T_{jk}) = \text{var}(T) - \text{var}(e_{ij})$

Then, we can re-estimate the $\pi_{jk}$ by updating
\[
\hat{\pi}_{jk}^{(n+1)} = \frac{\sum_{i=1}^C \mathbb{P}(e_{ij} = k | r_{ij}, \{\pi_{jk}^{(n)}\})}{C}
\]

{\bf TODO: incorporate gene length. This might change the $C_k$ constant, so check that...}
\end{document}  